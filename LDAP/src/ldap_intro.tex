Bestandssystemen gebruiken nummers om bestanden te koppelen aan gebruikers. Bij een bestand wordt de User-ID en de Group-ID opgeslagen om te bepalen welke gebruikers en groepen toegang hebben tot een bestand. Als we een bestandssysteem willen delen over het netwerk om data met elkaar te kunnen delen dan moeten alle systemen die bij de gedeelde data kunnen weten welke UID of GID bij welke gebruikers en groepen horen. We kunnen natuurlijk proberen om dat op elk systeem handmatig op orde te maken, maar het zou makkelijker zijn als we een centrale database hebben waar elke machine aan kan vragen wat er bij een UID of GID hoort.

Een mogelijke centrale netwerk oplossing die heel veel gebruikt wordt is LDAP\index{LDAP}. De afkorting LDAP staat voor Lightweight Directory Access Protocol\index{Lightweight Directory Access Protocol}. Het Lightweight deel staat voor het feit dat het een uitgeklede versie is van het ITU X.500 protocol. X.500, afkomstig uit de telefoniewereld, kent het DAP (Directory Access Protocol) om data uit het systeem te halen. DAP is heel uitgebreid en dus arbeidsintensief, vandaar dat er voor gebruik over TCP/IP netwerken een lichter alternatief werd bedacht: LDAP.

LDAP is een netwerk protocol dat beschrijft hoe data aangeboden moet worden aan het LDAP systeem. De data die uit het systeem verkregen worden of erin gestopt worden zijn in het LDIF formaat.


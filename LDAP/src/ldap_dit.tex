LDAP\index{LDAP} is een database met informatie over een gebruiker, zoals de gebruikersnaam, de groepen waarvan de gebruiker lid is en de User ID, maar ook gegevens als telefoonnummers, e-mail adressen en andere gegevens van een gebruiker kunnen worden opgenomen in LDAP. We kunnen LDAP dan ook vergelijken met een adresboek (telefoonboek). In LDAP kan ook data worden opgeslagen over computers en groepen. Kortom LDAP is eigenlijk een manier om data gestructureerd op te slaan. De structuur is vergelijkbaar met een bestandssysteem, een hi\"erarchische manier, met mappen en submappen waarin data staat. Vandaar de naam Directory\index{Directory}\index{LDAP!Directory} in LDAP.

Een \textquote{map} in LDAP zou een gebruiker kunnen zijn waaronder de gegevens van de gebruiker worden opgeslagen. Een map kan ook een bedrijf zijn met daarin de bedrijfsgegevens, zoals adressen, met daaronder mappen voor werknemers met daarin de gebruikersgegevens en contactgevens van deze medewerkers. Een LDAP-tree kan dus op verschillende manieren gestructureerd zijn. Via zoekopdrachten kan er in deze databoom gezocht worden. Binnen LDAP heten de mappen objecten\index{Object}. Een object kan andere objecten bevatten.

LDAP is een hi\"erarchische tree van opgeslagen data. Via zogenaamde schema bestanden wordt bepaald wat er in een object in de boom opgeslagen kan, mag of moet worden. Een schema bestand zegt dus wat bij bijvoorbeeld een user object er aan data aanwezig moet zijn en welke data er ook aanwezig mag zijn. De gebruikersnaam is verplicht, net als een UID, een telefoonnummer is niet verplicht. Dit geldt voor een LDAP systeem dat is ingericht met schema bestanden voor gebruik als gebruikersdatabase.

Omdat elke organisatie zijn eigen LDAP-tree kan hebben met zijn eigen gebruikers (zijn eigen interne adresboek) en LDAP systemen theoretisch wereldwijd te koppelen zijn is het handig om de basis van de tree (de root) aan te laten sluiten bij wereldwijd unieke naam van de organisatie. De keuze binnen de meeste organisaties is gevallen op het aansluiten bij het DNS systeem, daar dit het enige systeem is dat een unieke naam heeft voor een organisatie namelijk zijn domeinnaam. Er kan wereldwijd maar \'e\'en microsoft.com zijn.

LDAP werk niet met domeinnamen, maar met objecten. Waarbij het eerst genoemde object de root. Zo zou de basis van de LDAP tree voor Microsoft DC=microsoft,\allowbreak DC=com kunnen zijn. DC\index{DC} staat voor Domain Component\index{Domain Component} en verwijst naar het feit dat deze LDAP tree is opgebouwd volgens het DNS systeem. In de notatie valt op dat het aansluit op de DNS structuur met een beschrijving van elk object in de tree. Zo zouden we onze gebruikers kunnen stoppen in een organisatorische eenheid ofwel een Organizational Unit\index{Organizational Unit}. Laten we daarvoor de LDAP eenheid nemen: \index{OU}OU=users,\allowbreak DC=microsoft,\allowbreak DC=com. Daaronder kunnen we onze gebruikers stoppen CN=bgates,\allowbreak OU=users,\allowbreak DC=microsoft,\allowbreak DC=com. CN\index{CN} staat hierbij voor Common Name\index{Common Name}. Sommige organisaties gebruiken voor de Common Name de loginnaam, andere gebruiken de volledige naam van de gebruiker of het nummer van werknemer. Het is belangrijk dat de naam van elk object binnen een tak (branch) van de boom uniek is. Er kan dus niet nog een CN=bgates voorkomen binnen OU=users,\allowbreak DC=microsoft,\allowbreak DC=com. Een object in de boom die andere objecten kan bevatten worden ook wel Containers\index{Container} genoemd, in de Windows wereld wordt met Container vaak een Common Name object bedoeld.

Het \textquote{pad} CN=bgates,\allowbreak OU=users,\allowbreak DC=microsoft,\allowbreak DC=com is een unieke entry van een object in LDAP. Dit noemen we de Distinguished Name\index{Distinguished Name}. De Distinguished Name is het volledige pad naar een object in de tree.

De volledige LDAP data boom van een organisatie noemen we een DIT\index{DIT} (Directory Information Tree\index{Directory Information Tree}). Een DIT bevat zoals gezegt objecten\index{Object} en objecten hebben Attributes\index{Attribute}. Een attribute kan bijvoorbeeld het telefoonnummer zijn.


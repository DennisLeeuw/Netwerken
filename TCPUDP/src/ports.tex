Een van de taken van de transport-layer is om te zorgen dat de verstuurde data naar de juiste ontvanger wordt gestuurd. Bij de post is dat de geadresseerde, of de ontvanger. Meestal staat er een naam van een persoon. Ook onze systemen kennen bewoners die bepaalde taken uitvoeren. Zo kan op een server een web-server en een mail-server. Dus als we iets aan die server willen sturen zullen we moeten bepalen of dat aan de mail-server is of aan de web-server. Computers kunnen slecht omgaan met namen, maar zijn heel goed in getallen. Dus we gebruiken voor de diensten (web-server, mail-server) die op een machine draaien dan ook nummers, de zogenaamde port-numbers. Elke dienst (service) heeft zijn eigen port.

Mail gebruikt port 25 (SMTP) en de web-server gebruikt port 80 (HTTP). Tussen haakjes staat de naam van de service zoals wij mensen die kennen, want wij zijn beter met namen dan met getallen.


\documentclass[a4paper,12pt,twoside,openright,titlepage]{book}

%Additional packages
\usepackage[utf8]{inputenc}
\usepackage[T1]{fontenc}
\usepackage[dutch,english]{babel}
\usepackage{syntonly}
\usepackage[official]{eurosym}
\usepackage{graphicx}
\graphicspath{ {./images/} }
\usepackage{float}
\usepackage{xurl}
\usepackage{hyperref}
\hypersetup{colorlinks=true, linkcolor=blue, citecolor=blue, filecolor=blue, urlcolor=blue, pdftitle=, pdfauthor=, pdfsubject=, pdfkeywords=}
\usepackage{multirow,tabularx}
\usepackage[table]{xcolor}
\usepackage{scrextend}
\addtokomafont{labelinglabel}{\sffamily}
\usepackage{listings}
\usepackage{adjustbox}
\usepackage{csquotes}

%define inch
\usepackage{mathpazo,amsmath}
\def\inch#1{#1''}

% Turn on indexing
\usepackage{imakeidx}
\makeindex[intoc]

% Define colors
\usepackage{color}
\definecolor{ashgrey}{rgb}{0.7, 0.75, 0.71}

% Listing style
\lstset{
  backgroundcolor=\color{ashgrey},   % choose the background color; you must add \usepackage{color} or \usepackage{xcolor}; should come as last argument
  basicstyle=\footnotesize,        % the size of the fonts that are used for the code
  breakatwhitespace=false,         % sets if automatic breaks should only happen at whitespace
  breaklines=true,                 % sets automatic line breaking
  extendedchars=true,              % lets you use non-ASCII characters; for 8-bits encodings only, does not work with UTF-8
  frame=single,	                   % adds a frame around the code
  keepspaces=true,                 % keeps spaces in text, useful for keeping indentation of code (possibly needs columns=flexible)
  rulecolor=\color{black},         % if not set, the frame-color may be changed on line-breaks within not-black text (e.g. comments (green here))
  showspaces=false,                % show spaces everywhere adding particular underscores; it overrides 'showstringspaces'
}

% Uncomment for production
% \syntaxonly

% Style
\pagestyle{headings}

%%%%%%%%%%%%%%%%%%
% Begin document %
%%%%%%%%%%%%%%%%%%

% Define document
\author{D. Leeuw}
\title{TCP en UDP}
\date{\today\\v.0.5.0}

\begin{document}
\selectlanguage{dutch}

\maketitle

\copyright\ 2023 Dennis Leeuw\\

\begin{figure}[H]
\includegraphics[width=0.3\textwidth]{CC-BY-SA-NC.png}
\end{figure}

\bigskip

\input{src/CC-licentie}

%%%%%%%%%%%%%%%%%%%
%%% Introductie %%%
%%%%%%%%%%%%%%%%%%%

\frontmatter
\chapter{Over Dit Document}
%\input{src/DocOver_VPN}
%\input{src/DocChanges_VPN}
\section{Leerdoelen}
De lezer van dit document heeft na bestudering van de inhoud kennis van:
\begin{itemize}
\item wat kerberos is
\item welk probleem kerberos oplost
\item een globaal idee hoe de verschillende onderdelen van kerberos in elkaar zitten
\end{itemize}

\section{Voorkennis}
Voordat je aan deze les begint is basis kennis van ethernet wenselijk:
\begin{itemize}
\item Je moet weten wat een MAC-adres is.
\item Je moet weten hoe een ethernet frame eruit ziet.
\end{itemize}


%%%%%%%%%%%%%%%%%
%%% De inhoud %%%
%%%%%%%%%%%%%%%%%
\tableofcontents

\mainmatter

\chapter{Data integriteit}
Op ons lokale netwerk (LAN) kunnen we er redelijk vanuit gaan dat data ongeschonden aankomt, maar hoe weten we zeker dat als we data van de ene kant van de wereld naar de andere kant sturen dat onze data dan goed is aangekomen. Het zou toch heel vervelend zijn als we een e-mail versturen dat er dan enkele zinnen missen. Het controleren dat alle packetten die we verzonden hebben ook daadwerkelijk aangekomen zijn is de verantwoordelijkheid van de Transport-layer (OSI layer 4).

\section{TCP}
TCP is het transport protocol dat gebruikt wordt met IP. Vandaar dat we de protocol-stack kennen als TCP/IP. TCP staat voor Transport Control Protocol en het is verantwoordelijk voor het controleren van de verzonden en ontvangen data.

Een e-mail wordt niet in zijn geheel in een packet gestopt en dan over het netwerk gestuurd, maar het wordt opgehakt en allemaal kleine packetjes die stuk voor stuk verstuurd worden en aan de andere kant dus weer tot een e-mail verwerkt moeten worden. De reden achter dit systeem is heel simpel, stel iemand wil een film kijken over het netwerk en deze film duurt 1,5 uur. Als dan de film verstuurd zou worden als 1 packet dan kan er gedurende 1,5 uur niemand gebruik maken van het netwerk. Om dat te voorkomen wordt de film opgehakt in hele kleine stukjes die over het netwerk verstuurt worden. Tussen de kleine packetjes in kan dus ook iemand anders nog gebruik maken van het netwerk. Zo wordt het netwerk gedeeld, de data wordt gesegmenteerd en de data op laag 4 van het OSI-model heten dan ook segmenten.

Bij het versturen van segmenten moeten ze genummerd worden. Door ze te nummeren weten we of we alle segmenten hebben en of ze in de juiste volgorde staan. Het is aan TCP om dit allemaal te verzorgen.

Een andere taak van TCP is om de verzendende partij te laten weten dat alles goed is aangekomen. Voor elk ontvangen segment stuurt TCP een bevestiging terug. De verzendende partij weet dan dat het goed is aangekomen. Mocht het na een bepaalde tijd geen bevestiging van ontvangst krijgen dan kan het dat segment opnieuw versturen. Op deze manier weten we zeker dat de complete e-mail bij de ontvanger aankomt en dat er geen woorden of zinnen missen.

\section{UDP}
TCP is behoorlijk omslachtig. De verzendende partij moet in de gaten houden of het voor elk verzonden segment een ontvangstbevestiging heeft ontvangen en de ontvangende partij moet nagaan of alle data is aangekomen en in de juiste volgorde staat. Toch is dit de enige manier om data op de juiste manier over een onbetrouwbaar netwerk te versturen.

Maar wat nou als de integriteit van de data niet zo belangrijk is en dat snelheid bijvoorbeeld van veel groter belang is?

Laten weer de film van 1,5 uur nemen. Als we die op hakken in hele kleine segmenten, dan is het misschien niet zo heel erg als we een segment missen. Dat levert misschien een kleine vervorming op op het scherm, maar de film gaat gewoon verder. Als we zouden moeten wachten tot het segment opnieuw verstuurt is dan zou het een freeze van het scherm kunnen opleveren omdat we zitten te wachten op dat ene segment. Bij een film is het dus belangrijker dat de film doorgaat dan dat hij volledig elke pixel laat zien.

Voor dit soort data overdracht is UDP bedacht. UDP werk ook op de transport-laag, maar het controleert niet of de data correct is aangekomen en het kijkt ook niet of de segmenten wel in de juiste volgorde staan. Hogere lagen in de protocol-stack kunnen natuurlijk altijd nog deze functies uitvoeren, maar UDP doet het niet.

UDP wordt dan ook veel gebruikt als de integriteit van data niet zo belangrijk is, maar snelheid en reactietijd wel. Denk hierbij aan telefonie over Internet, videoconferencing, maar ook het spelen van netwerk-games.


\chapter{De eindbestemming}
Een van de taken van de transport-layer is om te zorgen dat de verstuurde data naar de juiste ontvanger wordt gestuurd. Bij de post is dat de geadresseerde, of de ontvanger. Meestal staat er een naam van een persoon. Ook onze systemen kennen bewoners die bepaalde taken uitvoeren. Zo kan op een server een web-server en een mail-server. Dus als we iets aan die server willen sturen zullen we moeten bepalen of dat aan de mail-server is of aan de web-server. Computers kunnen slecht omgaan met namen, maar zijn heel goed in getallen. Dus we gebruiken voor de diensten (web-server, mail-server) die op een machine draaien dan ook nummers, de zogenaamde port-numbers. Elke dienst (service) heeft zijn eigen port.

Mail gebruikt port 25 (SMTP) en de web-server gebruikt port 80 (HTTP). Tussen haakjes staat de naam van de service zoals wij mensen die kennen, want wij zijn beter met namen dan met getallen.



\chapter{Headers}
\section{TCP header}

\begin{tabular}{ |c|c|c|c|c|c| }
\hline
	\multicolumn{2}{|c|}{SPORT} & \multicolumn{2}{c|}{DPORT} \\
\hline
	\multicolumn{4}{|c|}{SEQ\#} \\
\hline
	\multicolumn{4}{|c|}{ACK\#} \\
\hline
	Length & Res & Flags & Window \\
\hline
	\multicolumn{2}{|c|}{Checksum} & \multicolumn{2}{c|}{Pointer} \\
\hline
	\multicolumn{2}{|c|}{Options} & \multicolumn{2}{c|}{Padding} \\
\hline
	\multicolumn{4}{|c|}{Data} \\
\hline
\end{tabular}

\begin{description}
	\item[SPORT] Source Port
	\item[DPORT] Destination Port
	\item[SEQ\#] Sequence Number
	\item[ACK\#] Acknowledgement Number
	\item[Length] Length of the header
	\item[RES] Reserved bits
	\item[FLAGS] Flags that van be set
	\item[Window] Sliding window-size
	\item[Checksum] TCP checksum
	\item[Pointer] Urgent pointer
	\item[Options] 0 of meer opties
	\item[Padding] 0 of meer padding bits
	\item[Data] Data afkomstig van hoger gelegen lagen
\end{description}


\section{UDP header}

\begin{tabular}{ |c|c|c|c|c|c| }
\hline
	SPORT & DPORT \\
\hline
	Length & Checksum \\
\hline
	\multicolumn{2}{|c|}{Data} \\
\hline
\end{tabular}

\begin{description}
	\item[SPORT] Source Port
	\item[DPORT] Destination Port
	\item[Length] Length of the header
	\item[Checksum] UDP checksum
	\item[Data] Data afkomstig van hoger gelegen lagen
\end{description}



\chapter{Opdrachten}
\section{nmap}

%\chapter{Praktijk opdrachten}
%\section{Router Configuratie}

%%%%%%%%%%%%%%%%%%%%%
%%% Index and End %%%
%%%%%%%%%%%%%%%%%%%%%
%\backmatter
\printindex
\end{document}

%%% Last line %%%

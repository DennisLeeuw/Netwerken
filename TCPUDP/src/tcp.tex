TCP is het transport protocol dat gebruikt wordt met IP. Vandaar dat we de protocol-stack kennen als TCP/IP. TCP staat voor Transport Control Protocol en het is verantwoordelijk voor het controleren van de verzonden en ontvangen data.

Een e-mail wordt niet in zijn geheel in een packet gestopt en dan over het netwerk gestuurd, maar het wordt opgehakt en allemaal kleine packetjes die stuk voor stuk verstuurd worden en aan de andere kant dus weer tot een e-mail verwerkt moeten worden. De reden achter dit systeem is heel simpel, stel iemand wil een film kijken over het netwerk en deze film duurt 1,5 uur. Als dan de film verstuurd zou worden als 1 packet dan kan er gedurende 1,5 uur niemand gebruik maken van het netwerk. Om dat te voorkomen wordt de film opgehakt in hele kleine stukjes die over het netwerk verstuurt worden. Tussen de kleine packetjes in kan dus ook iemand anders nog gebruik maken van het netwerk. Zo wordt het netwerk gedeeld, de data wordt gesegmenteerd en de data op laag 4 van het OSI-model heten dan ook segmenten.

Bij het versturen van segmenten moeten ze genummerd worden. Door ze te nummeren weten we of we alle segmenten hebben en of ze in de juiste volgorde staan. Het is aan TCP om dit allemaal te verzorgen.

Een andere taak van TCP is om de verzendende partij te laten weten dat alles goed is aangekomen. Voor elk ontvangen segment stuurt TCP een bevestiging terug. De verzendende partij weet dan dat het goed is aangekomen. Mocht het na een bepaalde tijd geen bevestiging van ontvangst krijgen dan kan het dat segment opnieuw versturen. Op deze manier weten we zeker dat de complete e-mail bij de ontvanger aankomt en dat er geen woorden of zinnen missen.

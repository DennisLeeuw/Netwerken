De hoeveelheid aangesloten netwerken op het Internet maakt het onmogelijk om handmatig alle netwerken met hun routers bij te houden. Om deze complexiteit op te lossen moeten we het beheren van de routing tabellen over laten aan geautomatiseerde processen. De meeste simpele vorm is dat elke router zijn routing tabel opstuurt naar de hem bekende routers en default gateways. Als elke router en gateway dat doet dan weten uiteindelijk alle routers op de wereld alle netwerken. Dit proces is min of meer zoals RIP werkt, het Routing Information Protocol.

Het gevolg van RIP is dat routers heel veel geheugen moeten hebben om alle informatie op te kunnen slaan. Er zijn in de loop van de tijd slimmere protocolen bedacht zoals OSPF en BGP om het wereldwijde routing probleem op te lossen.

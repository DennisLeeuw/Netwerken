Voor moderne netwerken wordt bijna altijd het OSI-model gebruikt om duidelijk te maken wat er binnen een protocol-stack gebeurd. Het meest gebruikte netwerk protocol, TCP/IP, is echter veel ouder dan het OSI-model en TCP/IP kent dan ook zijn eigen model waarop het gebaseerd is. Dit model staat bekent als het ARPA-model.

het ARPA-model kent maar 4 lagen, namelijk de Application-layer, Host-to-Host-layer, Internet-layer en de Network-Interface-layer. Met een beetje goede wil kunnen we deze lagen ongeveer overeen laten komen met de lagen uit het OSI-model.

De Network-Interface is alles dat te maken heeft met de fysieke laag en de netwerkkaarten. De Internet laag zorgt voor de adressering van de netwerk (IP) pakketten. De Host-to-Host laag is verantwoordelijk voor het verzenden van de data in de juiste grote en de controle van de data op het juist aankomen. De Applicatie laag is verder verantwoordelijk voor alle andere zaken die te maken hebben met het opmaken en verzenden of ontvangen van de data.

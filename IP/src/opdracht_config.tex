Elke computer op een TCP/IP netwerk, heeft minimaal \'e\'en IP-adres nodig. De verschillende operating systems hebben hun eigen manier om een machine te voorzien van een IP-adres. Er zijn twee methodes om dit te doen. Er is de handmatige manier en de automatische. Om automatisch een IP-adres te verkrijgen is er op het netwerk een DHCP-server nodig. De meeste bestaande netwerken zijn hiervan voorzien, dus als je al een IP-adres hebt dan is dit gekomen door de DHCP-server.

Mocht het IP-adres op je netwerkkaart beginnen met 169.254.x.y dan is dat een teken dat er geen DHCP-server beschikbaar is.

\begin{enumerate}
\item Controlleer wat het IP-adres op je systeem is
\item Zoek uit waar in je besturingssysteem de configuratie van het IP-adres moet gebeuren (zoek op Internet)
\item Zoek uit of de configuratie op automatisch of handmatig IP-adres staat
\item Als de configuratie op handmatig staat noteer dan alle gegevens die nu geconfigureerd zijn
\item Als de configuratie op automatisch (DHCP) staat zet deze dan op handmatig
\item Geef je netwerkkaart het IP adres 192.168.42.42 en een subnet-mask van 255.255.255.255
\item Controlleer wat het IP-adres op je systeem is
\item Als je klaar bent zet dan alles terug naar de oorspronkelijke stand
\item Controlleer wat het IP-adres op je systeem is en of dat overeenkomt met wat je bij punt 1 van deze opdracht hebt aangetroffen
\item Wat is het IP adres van de default gateway op je machine? (zoek op Internet hoe je dat uitzoekt)
\end{enumerate}


Een andere tool die we kunnen gebruiken om een IP-netwerk te testen en die vooral handig is voor het testen of we een Internet verbinding hebben en hoe die verbinding loopt is het traceroute commando. Traceroute doet zoals de naam als zegt het weergeven van de route van een packet. Op een Linux en Mac OS X heet het commando \texttt{traceroute} op Windows heet het \texttt{tracert}

\begin{enumerate}
\item Type op de commandline: \texttt{traceroute 8.8.8.8}
\item Het kan zijn dat de output op een regel een of meerdere sterren (*) geeft. Dat is niet erg het geeft aan dat een tussenliggende router geen antwoord wil geven,
\item De hulp van de tool kan je op Windows opvragen met \texttt{tracert /?} op Linux en Mac OS X met \texttt{traceroute -h}
\item Welke optie kan je gebruiken om geen address resolving te doen?
\end{enumerate}

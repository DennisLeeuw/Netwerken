Het is ondoenlijk om elke computer in het netwerk te voorzien van routing tabellen van het hele Internet. Het is makkelijker om \'e\'en centrale machine in het netwerk op te hangen die de weg naar het Internet weet. Als een laptop dan bijvoorbeeld een packet heeft voor een ander netwerk dan verstuurt het dat naar die centrale machine en die centrale machine lost het op. Zo'n centrale machine heet een default gateway. Dat kan een router zijn, maar ook een firewall of een proxy server. Voor ons is de verdere functie van de machine niet van belang. We hebben een default gateway ofwel een apparaat waar we data naar kunnen sturen als we het niet meer weten.

Een computer die geen default gateway heeft ingesteld kan dus alleen data versturen naar zijn direct aangesloten netwerk. Hij kan niet communiceren met andere netwerken en al helemaal niet met andere machines die aangesloten zijn op het Internet.

Je kan de default gateway vergelijken met de brievenbus van de post. Ik heb geen idee hoe de post ervoor zorgt dat mijn brief in Canada terecht komt. Ik weet wel dat als ik de brief in de oranje brievenbus van de post gooi, dat zij hun best doen om te zorgen dat hij in Canada komt. Zo werkt dat ook met de default gateway. Ik weet niet hoe die het doet, maar hij gaat zijn best doen om te zorgen dan mijn packet bij bijvoorbeeld Google uitkomt.

Wat we kunnen doen voor computer kunnen we natuurlijk ook doen op routers. Een router kent zijn direct aangesloten netwerken, maar als hij het niet meer zou een default gateway voor een router ook wel makkelijk zijn en tot op zekere hoogte kan dat. Helaas gaat dat bij de complexiteit van het Internet niet meer op.

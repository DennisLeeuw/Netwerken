Ethernet is ooit bedacht om data over een kabel te kunnen versturen. Om te zorgen dat iedereen op de kabel elkaar kan verstaan is het noodzakelijk om allemaal dezelfde taal te spreken. De collectieve taal voor ethernet heet CSMA/CD, waarop we later terug zullen komen.

In 1973 stuurde Robert Metcalfe een memo aan zijn bazen van Xerox Parc over de mogelijkheden van ethernet. Het idee werd verder ontwikkeld wat uiteindelijk het document \textquote{Ethernet: Distributed Packet-Switching For Local Computer Networks} opleverde dat in 1976 werd gepubliceerd door Robert Metcalfe en David Boggs.

Metcalfe verliet in 1979 Xerox om 3Com op te richten. Het doel was om LAN's populair te maken, hiervoor zocht hij de samenwerking op met Dec, Intel en zijn oud werkgever Xerox. Met dit trio werd de DIX-standaard gepubliceerd zodat netwerk componenten van deze verschillende bedrijven met elkaar konden samenwerken. De formele standaard voor ethernet kwam er in 1983 toen de IEEE de IEEE 802.3 standaard publiceerde. Met deze laatste publicatie was er een offici\"ele werelwijde standaard.

De IEEE802.3 kent allerlei subsecties die worden aangegeven met een letter. Zo is 100 Mbit/s ethernet over twisted-pair kabel IEEE802.3u. We zullen hier verder niet op ingaan, maar het is wel handig om te weten dat er voor elke snelheid en elk medium (coax, twisted-pair, glas) een standaard is die beschrijft waaraan zowel het medium als de aangesloten netwerkkaarten moeten voldoen (Wikipedia heeft een overzicht van de verschillende standaarden: \url{https://en.wikipedia.org/wiki/IEEE_802.3})

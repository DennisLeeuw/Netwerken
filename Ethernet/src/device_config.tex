Om bridges en switches zijn soms op afstand beheerbaar. Een ouderwets protocol dat gebruikt werd om ze te beheren is telnet, modernere protocollen zijn ssh en natuurlijk de web-interface. Om te een switch of een bridge remote te kunnen beheren moeten we kunnen inloggen op het apparaat. We moeten dus een gebruikersnaam en wachtwoord kunnen opgeven en deze data wordt over het netwerk verzonden. Als iemand het netwerk kan afluisteren dan heeft hij dus onze gebruikersnaam en wachtwoord en dat is iets dat we niet willen. Dit geldt voor telnet en http, de veilige protocollen zijn ssh en https. Het https en ssh protocol encrypt alle data voordat die over het netwerk gaat zodat afluisteraars niets met de ontvangen data kunnen.

Als je dus een nieuwe switch of bridge in je netwerk plaatst en deze kan remote beheerd worden, zorg er dan als eerste voor dat je de onveilige protocollen uit zet, de veilige protocollen aan zet en het wachtwoord op de machine wijzigt.

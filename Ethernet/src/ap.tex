Een Access Point of AP is een bridge die de verbinding legt tussen het draadloze netwerk en een bedraad ethernet netwerk.

Moderen varianten van de Access Point kunnen ook de koppeling naar Internet maken en ondersteunen naast ethernet ook het TCP/IP protocol (het protocol om via IP adressen te communiceren met machines over de gehele wereld). Deze slimmere apparaten hebben vaak ook een firewall en een DHCP-server.

De meest eenvoudige en oorspronkelijke vorm van het AP was een apparaat met de bridge functionaliteit en dat is nog steeds de basis van een AP.


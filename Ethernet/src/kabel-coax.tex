Ethernet is ooit ontworpen om gebruikt te worden over coax-kabel. Alle aangesloten netwerkkaarten waren fysiek aangesloten op dezelfde kabel. Er was \'e\'en kabel met daarop meerdere netwerkkaarten die de kabel met elkaar moesten delen. Het oorspronkelijke design was op een hele dikke gele coax-kabel, later werd dit een veel dunnere RG-58 kabel waarbij de kabel via bajonet connectoren aangesloten werd op de netwerkkaart.

Een ethernet-kabel met de aangesloten netwerkkaarten mocht maximaal 100 meter lang zijn, omdat anders het signaal te veel zou verzwakken. De kabel met aangesloten netwerkkaarten vormt het collision domain.


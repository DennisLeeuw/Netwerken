In deze opdracht gaan we onze eigen UTP-patch-cable maken. De huidige ethernet standaarden voor ethernet over UTP-kabel zeggen dat de totale kabellengte vanaf de switch naar de netwerkkaart 100 meter lang mag zijn. Van deze 100 meter mag er 2x 5 meter getwijnde twisted pair koperkabel zijn, de zogenaamde patch-kabels. De rest van de kabe moet solid koper zijn.

De solid koper kabel is een acht-aderige kabel waarbij elke ader van massief koper is. Deze kabel wordt gebruikt om patch-panelen te verbinden met wall-outlets op de verschillende kamers van een gebouw. De kabel ligt meestal door kabelgoten.

Patch kabels zijn gemaakt van getwijnt koper (Engels: stranded copper). Dat betekent dat elke draad van de acht-aderige kabel, bestaat uit allemaal fijne draadjes. Deze opbouw zorgt ervoor dat de kabel flexibel is en makkelijk in bochtjes gebogen kan worden. De patch-kabel wordt gebruikt om de patch-panelen aan te sluiten op de switch en om werkstations aan te sluiten op de wall-outlet op de werkkamer.

De solid-koper kabels worden vaak gelegd door installatie bedrijven en patch-kabels worden meestal ingekocht, maar het kan weleens gebeuren dat je even snel een eigen patch-kabel moet maken. Dat is wat we in deze opdracht gaan oefenen. Wat heb je nodig:
\begin{itemize}
\item Een stuk stranded koper kabel (1 meter)
\item 2 RJ45 connectoren voor stranded koper
\item Een RJ45 krimptang
\item Een kabel stripper
\item Een kabelkniptang
\item Een kabel-tester
\end{itemize}

Alle aders binnen een 8-aderige kabel hebben hun eigen kleur en moeten op de juiste manier in de RJ45 connector terecht komen. De juiste manier voor ethernet kan op twee manieren namelijk via de EIA/TIA-568-A of -B standaard. In Europa gebruiken we bijna allemaal de B-standaard. Zoek op Internet op wat de volgorde van de kabel moet zijn in de RJ45 connector (zoeken op plaatjes met de term TIA-568-B).

Volg de aanwijzingen op \url{https://www.ditecpro.be/data/document/311/1561390222-UTP-kabel-maken.pdf} om je eigen kabel te maken. Let op dat je aan beide zijden van de kabel een connector zet.

Test je kabel op een correcte werking. Dit kan door hem in een netwerk te gebruiken, door gebruik te maken van een simpele kabel-tester (deze controleert alleen of de aders op de juiste plek zijn aangesloten), je kan ook een kwaliteitstest doen, dan heb je een duurdere tester nodig, de controleert of de kabel ook voor de geleiding aan alle kwaliteitseisen voldoet.

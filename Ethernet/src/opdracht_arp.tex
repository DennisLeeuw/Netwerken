Het Internet gebruikt een protocol dat TCP/IP heet. TCP/IP op een lokaal bedraad en draadloos netwerk (LAN - Local Area Network) gebruikt ethernet adressen om frames lokaal af te leveren. Om uit te zoeken welke adressen er bij welke machine horen wordt er gebruik gemaakt van het ARP-protocol. Het protocol heeft ook een commando genaamd \texttt{arp} dat je gebruiken om te zien welke machines jouw machine kent. Gebruik om je machine, afhankelijk van je besturingssysteem het volgende commando:
\begin{description}
\item[Windows] arp -a
\item[Linux] arp -a
\item[Mac OS X] arp -a
\end{description}

In de voorgaande opdracht heb je een MAC-adres genoteerd van een destination waar je machine mee communiceerde. Zoek dat MAC-adres op de de arp-tabel die je zojuist hebt opgevraagd en noteer welke machine (naam en/of IP-adres) daarbij hoorde.

Je zal je misschien afvragen waarom in een document over ethernet ook WiFi wordt behandeld. Het antwoord is vrij eenvoudig: omdat WiFi ethernet is met een paar aanpassingen.

Ook WiFi gebruikt MAC-addressen en ook het gebruikte protocol is bijna hetzelfde als ethernet. Er is bewust gekozen voor deze gelijkenis. Tijdens het ontwerp van WaveLAN, de voorloper van WiFi, was ethernet al een veel gebruikte standaar, hierdoor waren de chips voor de interfaces goedkoop. Door bij ethernet aan te sluiten werd er gekozen voor relatief goedkope chips en bij bestaande kennis.

WiFi is ontwikkeld onder de naam WaveLAN door NCR Systems Engineering in Nieuwegein gedurende 1986-1987. In 1991 werd het gestandardiseerd door de IEEE als IEEE802.11 en werd het bekend onder de naam Wi-Fi.

De naam WaveLAN werd nog langer gebruikt als de productnaam waaronder NCR, later AT\&T en weer later Lucent Technologies producten die voldeden aan de IEEE-standaard op de markt brachten.

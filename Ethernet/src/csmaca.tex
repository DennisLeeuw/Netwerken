Het protocol dat door WiFi gebruikt wordt om data door de lucht te versturen is CSMA/CA, Carrier Sense Multiple Access with Collision Avoidance. Het is een mondvol, het lijkt enorm veel op het ethernet protocol. Er wordt geluisterd of er geen signaal in de lucht is, als de "carrier" vrij is mag iedereen gaan zenden, maar om collisions te voorkomen (avoidance) wordt er eerst gebruik gemaakt van de back-off-time. Doordat er standaard gebruik wordt gemaakt van een back-off is WiFi iets trager dan ethernet, maar is de kans op collisions vele malen kleiner.

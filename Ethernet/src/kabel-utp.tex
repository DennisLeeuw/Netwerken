Later ontstond een ontwikkeling naar UTP-kabel. Het netwerk kreeg daardoor een ster-structuur met een hub (naaf/as) in het midden. Een hub is een actief electronisch apparaat die werkt als een repeater. Een repeater zal het binnenkomende signaal weer netjes maken en daarna doorsturen naar elke poort die hij heeft. Er kan dus vanaf elke poort van een hub een kabel gelegd worden van 100 meter, wat een spanweidte van het netwerk oplevert van 200 meter. Omdat een hub het binnenkomende signaal van een poort netjes maakt en naar elke andere poort doorstuurt (repeat) vormt de hub met de aangesloten netwerkkaarten een collision domain. Let op: ook de collisions worden gerepeat.

Met DHCP kan je meer uitdelen dan alleen IP-adressen. Standaard wordt er en IP-adres, een subnet-mask en een default gateway uitgedeeld. In de meeste netwerken wordt ook aangegeven wat de IP-adressen van de name-servers zijn.

Het is ook toegestaan om bijvoorbeeld de hostname, de domainname of extra routes mee te geven.

Totslot zijn er nog opties die het mogelijk te maken om de client te vertellen welke NTP-server of welke mail-server er gebruikt kan worden. Er zijn nog veel meer opties die uitgedeeld kunnen worden, maar deze laatste worden maar weinig gebruikt in de huidige netwerken.


\documentclass[a4paper,12pt,twoside,openright,titlepage]{book}

%Additional packages
\usepackage[ascii]{inputenc}
\usepackage[T1]{fontenc}
\usepackage[dutch,english]{babel}
\usepackage{syntonly}
\usepackage[official]{eurosym}
%\usepackage[graphicx]
\usepackage{graphicx}
\graphicspath{ {./images/} }
\usepackage{float}
\usepackage{xurl}
\usepackage{hyperref}
\hypersetup{colorlinks=true, linkcolor=blue, citecolor=blue, filecolor=blue, urlcolor=blue, pdftitle=, pdfauthor=, pdfsubject=, pdfkeywords=}
\usepackage{tabularx}
\usepackage{scrextend}
\addtokomafont{labelinglabel}{\sffamily}
\usepackage{listings}
\usepackage{adjustbox}
\usepackage{color}

%define inch
\usepackage{mathpazo,amsmath}
\def\inch#1{#1''}

% Turn on indexing
\usepackage{imakeidx}
\makeindex

% Define colors
\definecolor{ashgrey}{rgb}{0.7, 0.75, 0.71}

% Listing style
\lstset{
  backgroundcolor=\color{ashgrey},   % choose the background color; you must add \usepackage{color} or \usepackage{xcolor}; should come as last argument
  basicstyle=\footnotesize,        % the size of the fonts that are used for the code
  breakatwhitespace=false,         % sets if automatic breaks should only happen at whitespace
  breaklines=true,                 % sets automatic line breaking
  extendedchars=true,              % lets you use non-ASCII characters; for 8-bits encodings only, does not work with UTF-8
  frame=single,	                   % adds a frame around the code
  keepspaces=true,                 % keeps spaces in text, useful for keeping indentation of code (possibly needs columns=flexible)
  rulecolor=\color{black},         % if not set, the frame-color may be changed on line-breaks within not-black text (e.g. comments (green here))
  showspaces=false,                % show spaces everywhere adding particular underscores; it overrides 'showstringspaces'
}

% Uncomment for production
% \syntaxonly

% Style
\pagestyle{headings}

%%%%%%%%%%%%%%%%%%
% Begin document %
%%%%%%%%%%%%%%%%%%

% Define document
\author{D. Leeuw}
\title{Network Attached Storage}
\date{\today\\v.0.2.0}

\begin{document}
\selectlanguage{dutch}

\maketitle

\copyright\ 2021-2024 Dennis Leeuw\\

\begin{figure}
\includegraphics[width=0.3\textwidth]{CC-BY-SA-NC.png}
\end{figure}

\bigskip

\input{src/CC-BY-SA-NC}

%%%%%%%%%%%%%%%%%%%
%%% Introductie %%%
%%%%%%%%%%%%%%%%%%%

\frontmatter
\chapter{Over dit Document}
\section{Leerdoelen}
De lezer van dit document heeft na bestudering van de inhoud kennis van:
\begin{itemize}
\item wat kerberos is
\item welk probleem kerberos oplost
\item een globaal idee hoe de verschillende onderdelen van kerberos in elkaar zitten
\end{itemize}

\section{Voorkennis}
Voordat je aan deze les begint is basis kennis van ethernet wenselijk:
\begin{itemize}
\item Je moet weten wat een MAC-adres is.
\item Je moet weten hoe een ethernet frame eruit ziet.
\end{itemize}


%%%%%%%%%%%%%%%%%
%%% De inhoud %%%
%%%%%%%%%%%%%%%%%
\tableofcontents

\mainmatter
\chapter{Inleiding}

\chapter{Bestanden delen over het netwerk}
\section{FTP}
\section{HTTP/HTTPs}

\chapter{Network Filesystems}
\section{NFS - Network File System}\index{NFS}
\input{src/nfs}
\subsection{Security}
\input{src/nfs-security}
\subsection{Interoperability}
\input{src/nfs-interoperability}
\section{SMB - Server Message Block}\index{SMB}
\input{src/smb}
\subsection{Security}
\input{src/smb-security}
\subsection{Interoperability}
\input{src/smb-interoperability}
\subsection{Roaming profiles}\index{Roaming profiles}

\chapter{Network Fileserver}
\section{NAS - Network Attached Storage}
\input{src/nas}
\subsection{Opdracht - TrueNAS}
\input{src/nas_opdracht}
\input{src/fileserver}
\section{Failover in a Heartbeat}
\input{src/fs-failover}

\chapter{Web storage}\index{Web storage}
\input{src/webstorage}
\section{webdav}\index{Webdav}
\input{src/webdav}
\section{Amazon S3 bucket}\index{Amazon S3 bucket}
\section{Google Drive}\index{Google drive}
\section{One Drive \& Microsoft Azure BLOB}\index{One drive}\index{Azure BLOB}

\chapter{Object Storage}\index{Object storage}
\input{src/objectstorage}

%%%%%%%%%%%%%%%%%%%%%
%%% Index and End %%%
%%%%%%%%%%%%%%%%%%%%%
%\backmatter
\printindex
\end{document}

%%% Last line %%%

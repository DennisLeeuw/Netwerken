De tijd wordt bijgehouden door atoomklokken. Het is onhandig om een atoomklok in te bouwen in elke computer. De meest eenvoudige oplossing is om de tijd uit te zenden via de lucht en deze weer op te vangen met een ontvanger. Dat gebeurd met de DCF77 Tijdseinzender\index{DCF77 Tijdseinzender} in Duitsland. Wekkers, stationsklokken en elk ander apparaat kan dit signaal ontvangen en zorgen dat zijn interne klok gesynchroniseerd is met deze klok. Een ontvanger voor deze zender inbouwen in een computerkast is lastig want de standaard kasten zijn van staal en ontvangen dus geen radiosignalen van buitenaf, dat is precies de reden waarom ze van staal zijn om verstoring van buitenaf te minimaliseren. Er zou natuurlijk een ontvanger ingebouwd kunnen worden met een antenne buiten de kast, dat is een mogelijke oplossing en die oplossingen bestaan ook.

Daar tegenwoordig bijna elke computer aan een netwerk hangt is het handiger om de tijd via het netwerk te versturen. We moeten dan natuurlijk wel rekening houden met de vertragingen op het netwerk en daar compensatie berekeningen op loslaten, maar met wat goede wil en een paar hele slimme programmeurs moet dat te doen zijn. De oplossing die we tegenwoordig kennen heet NTP\index{NTP} (Network Time Protocol\index{Network Time Protocol}).

Op het Internet zijn er tijdservers die op een manier de atoomkloktijd doorzetten naar het Internet. Als iedereen tegen deze servers aan zou praten dan zou er al snel niet voldoende bandbreedte zijn. Om dit probleem op te lossen, de oplossing voor dit probleem is het gebruik van zogenaamde NTP-pools\index{NTP-pool}\index{Pool}. Vele servers samen vormen een pool waarbinnen de tijd gesynchroniseerd wordt, door tegen zo'n pool te praten wordt de load op het netwerk verdeeld. Je kan zelf ook een NTP-server opzetten en je aansluiten bij zo'n pool zodat je onderdeel wordt van de oplossing. Er zijn natuurlijk wel enkele eisen waaraan je moet voldoen om mee te doen met zo'n pool. E\'en van die eisen is dat de server 24/7 beschikbaar moet zijn.

Tijdservers op het Internet kunnen hun klok synchroniseren met een atoomklok\index{Atoomklok}, met het DCF77\index{DCF77} tijdsignaal of met GPS\index{GPS}. Al deze bronnen leveren een accurate tijdsbron.


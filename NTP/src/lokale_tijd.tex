Je computer houdt de tijd bij\index{Computer!tijd}\index{Tijd!Computer}. Standaard is dat een functie van het CMOS\index{CMOS}, dus ook als de computer uit staat wordt de tijd bijgehouden. De klok van je computer is afhankelijk van het kristal\index{kristal}\index{Computer!kristal} op het moederbord. Elk kristal is uniek en heeft dus ook een unieke frequentie. Computers lopen na verloop van tijd niet meer gelijk met de tijd in de wereld. Dit geldt overigens niet alleen voor computers ook klokken lopen na verloop van tijd voor of achter. De vraag die ontstaat is: wat is dan die tijd in de wereld?

Tegenwoordig gebruiken we atoomklokken\index{Atoomklok} om de tijd van de wereld in uit te drukken. Zie voor de werking van een atoomklok deze link: \url{https://www.quantumuniverse.nl/hoe-werkt-een-atoomklok}.

We hebben afspraken gemaakt om te zorgen dat iedereen op tijd op zijn werk is. We hebben de aarde opgedeeld in tijdzones\index{Tijdzones}. Binnen een tijdzone is de tijd hetzelfde is. Zo is het in Nederland, Belgi\"e, Frankrijk en Spanje altijd op hetzelfde moment 12:00 uur. In het Verenigd Koninkrijk is het een uur later dan bij ons en in Rusland is het een uur vroeger dan bij ons, zij zitten in een andere tijdzone. Tijzones lopen van de Noordpool naar de Zuidpool en binnen zo'n zone houden we dezelfde tijd aan. De nul-lijn\index{Nul-lijn} voor de lengte graad ligt over Greenwich\index{Greenwich} in het Verenigd Koninkrijk. In Nederlands zitten we dus op +1 en in Rusland op +2 ten opzichte van die nul-lijn. Als we de andere kant op gaan van de nul-lijn dan komen we in New York en zitten we op -5 uur ten opzichte van de nul-lijn. Een factor die het hele verhaal nog iets complexer maakt is het gebruik van de zomer-\index{Zomertijd} en wintertijd\index{Wintertijd}.

Er is dan ook voor gekozen om \'e\'en universele tijd te hebben en daaruit de lokale tijd te berekenen. Die universele tijd wordt UTC\index{UTC} (Coordinated Universal Time\index{Coordinated Universal Time}). Als je wil weten waarom de afkorting afwijkt van de naam verwijzen we je graag naar de Wikipedia-pagina \url{https://en.wikipedia.org/wiki/Coordinated_Universal_Time}. De voorganger van UTC is GMT\index{GMT} ofwel Greenwich Mean Time\index{Greenwich Mean Time}.


Computers zijn goed met getallen, maar wij mensen zijn beter met namen. Op een netwerk dat alleen uit IP-adressen zou bestaan zouden wij al snel de weg kwijt zijn. Het is voor ons makkelijker om te onthouden dat we bij google.com moeten zijn dan bij 142.251.36.46. Dit is precies de reden dat er een Domain Name System (DNS) is. DNS vertaald domeinnamen naar IP-adressen. Zo kunnen wij de naam gebruiken in onze browser en kan de computer het IP adres gebruiken om onze packetten over het netwerk te sturen.


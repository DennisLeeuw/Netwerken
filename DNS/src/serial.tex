De database van de verschillende DNSen kan wijzigen. Het kan bijvoorbeeld gebeuren dat de web-server op een andere machine komt te draaien en dat dus het IP-adres wijzigt. Dit willen we in het name-server database wijzigen, zolang de andere name-servers aan ons systeem vragen wat het IP adres is gaat het goed, dan is de wijziging onmiddelijk, maar als een name-server een cache heeft aangelegd, dan zal die voorlopig zijn cache blijven gebruiken. Het kan dus enige tijd duren voordat een wijziging op het gehele Internet bekend is.

De manier om caches te laten verlopen is het updaten van de \textquote{serial}. De serial is een numeriek ID. Een hogere serial betekent een nieuwere versie. Voor de serial wordt meestal de datum + een versie nummer genomen: 2023120603. Deze serial staat voor 6 december 2023 de derde wijziging. Je kan zo maximaal 99 wijzigingen op een dag doen (01 tot 99).

Een DNS kan dus de serial opvragen om te weten te komen of het opnieuw een query moet doen naar een IP-adres of dat hij de huidige cache kan laten bestaan.

